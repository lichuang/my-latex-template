%!TEX program = xelatex

\documentclass[lang=cn,newtx,10pt,scheme=chinese]{codedump}

\title{\bfseries\sffamily
  Sample: Angular 6 快速入门\\
  \normalfont\zihao{-3}
  One framework. Mobile \& desktop.
}
\author{WisdomFusion \\ \faGithubAlt~ https://github.com/WisdomFusion}
\date{}


\begin{document}
\maketitle
\thispagestyle{empty}

\section{什么是 Angular?}
\label{newfeatures}

Angular 是一个开发平台。它能帮你更轻松的构建 Web 应用。Angular 集声明式模板、依赖注入、端到端工具和一些最佳实践于一身,为你解决开发方面的各种挑战。Angular 为开发者提升构建 Web、手机或桌面应用的能力。

\section{架构概览}

Angular 是一个用 HTML 和 TypeScript 构建客户端应用的平台与框架。Angular 本身使用 TypeScript 写成的。它将核心功能和可选功能作为一组 TypeScript 库进行实现,你可以把它们导入你的应用中。

Angular 的基本构造块是 NgModule,它为组件提供了编译的上下文环境。 NgModule 会把相关的代码收集到一些功能集中。Angular 应用就是由一组 \verb|NgModule| 定义出的。 应用至少会有一个用于引导应用的根模块,通常还会有很多特性模块。

\begin{itemize}
  \item 组件定义\textbf{视图}。视图是一组可见的屏幕元素,Angular 可以根据你的程序逻辑和数据来选择和修改它们。 每个应用都至少有一个根组件。
  
  \item 组件使用\textbf{服务}。服务会提供那些与视图不直接相关的功能。服务提供商可以作为依赖被注入到组件中, 这能让你的代码更加模块化、可复用,而且高效。
\end{itemize}

\section{\ttfamily @NgModule \sffamily{元数据}}

\verb|NgModule| 是一个带有 \verb|@NgModule| 装饰器的类。\verb|@NgModule| 装饰器是一个函数,它接受一个元数据对象,该对象的属性用来描述这个模块。其中最重要的属性如下。

\begin{itemize}
  \item \verb|declarations|(可声明对象表) —— 那些属于本 \verb|NgModule| 的组件、指令、管道。
  
  \item \verb|exports|(导出表) —— 那些能在其它模块的组件模板中使用的可声明对象的子集。
  
  \item \verb|imports|(导入表) —— 那些导出了本模块中的组件模板所需的类的其它模块。
  
  \item \verb|providers| —— 本模块向全局服务中贡献的那些服务的创建器。 这些服务能被本应用中的任何部分使用。(你也可以在组件级别指定服务提供商,这通常是首选方式。)
  
  \item \verb|bootstrap| —— 应用的主视图,称为根组件。它是应用中所有其它视图的宿主。只有根模块才应该设置这个 \verb|bootstrap| 属性。
\end{itemize}


下面是一个简单的根 NgModule 定义:

\begin{lstlisting}[language=TypeScript,caption={src/app/app.module.ts}]
import { NgModule }      from '@angular/core';
import { BrowserModule } from '@angular/platform-browser';
@NgModule({
  imports:      [ BrowserModule ],
  providers:    [ Logger ],
  declarations: [ AppComponent ],
  exports:      [ AppComponent ],
  bootstrap:    [ AppComponent ]
})
export class AppModule { }
\end{lstlisting}

\begin{lstlisting}[style=bashInputStyle]
cd my-app
ng serve --open
\end{lstlisting}

\begin{lstlisting}[style=bashOutputStyle]
blablabla
\end{lstlisting}

\section{boxes}

\noindent\verb|\begin{titledBox}{<title>} <content> \end{titledBox}|

\begin{titledBox}{HTTP/Console 内核}
  HTTP 内核继承自 \verb|Illuminate\Foundation\Http\Kernel| 类,该类定义了一个 \verb|bootstrappers| 数组,这个数组中的类在请求被执行前运行,这些 \verb|bootstrappers| 配置了错误处理、日志、检测应用环境以及其它在请求被处理前需要执行的任务。
\end{titledBox}

\noindent\verb|\begin{noteBox} <content> \end{noteBox}|

\begin{noteBox}
  HTTP 内核继承自 \verb|Illuminate\Foundation\Http\Kernel| 类,该类定义了一个 \verb|bootstrappers| 数组,这个数组中的类在请求被执行前运行,这些 \verb|bootstrappers| 配置了错误处理、日志、检测应用环境以及其它在请求被处理前需要执行的任务。
\end{noteBox}

\noindent\verb|\begin{importantBox} <content> \end{importantBox}|

\begin{importantBox}
  HTTP 内核继承自 \verb|Illuminate\Foundation\Http\Kernel| 类,该类定义了一个 \verb|bootstrappers| 数组,这个数组中的类在请求被执行前运行,这些 \verb|bootstrappers| 配置了错误处理、日志、检测应用环境以及其它在请求被处理前需要执行的任务。
\end{importantBox}

\noindent\verb|\begin{tipBox} <content> \end{tipBox}|

\begin{tipBox}
  HTTP 内核继承自 \verb|Illuminate\Foundation\Http\Kernel| 类,该类定义了一个 \verb|bootstrappers| 数组,这个数组中的类在请求被执行前运行,这些 \verb|bootstrappers| 配置了错误处理、日志、检测应用环境以及其它在请求被处理前需要执行的任务。
\end{tipBox}

\noindent\verb|\begin{warningBox} <content> \end{warningBox}|

\begin{warningBox}
  HTTP 内核继承自 \verb|Illuminate\Foundation\Http\Kernel| 类,该类定义了一个 \verb|bootstrappers| 数组,这个数组中的类在请求被执行前运行,这些 \verb|bootstrappers| 配置了错误处理、日志、检测应用环境以及其它在请求被处理前需要执行的任务。
\end{warningBox}

\noindent\verb|\begin{shellBox} <content> \end{shellBox}|

\begin{shellBox}
cd my-app
ng serve --open
\end{shellBox}

\noindent\verb|\begin{invertedShellBox} <content> \end{invertedShellBox}|

\begin{invertedShellBox}
cd my-app
ng serve --open
\end{invertedShellBox}

\end{document}
